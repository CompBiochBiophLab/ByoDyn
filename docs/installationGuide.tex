% $Id: installationGuide.tex,v 4.14 2008/12/17 00:55:44 alglomana Exp $
\documentclass[a4paper, 11pt]{article}
\marginparwidth 0pt
\oddsidemargin  0pt
\evensidemargin  0pt
\marginparsep 0pt
\topmargin   0pt
\textwidth   6.5in
\textheight  8.5 in
\usepackage{url}
\usepackage{hyperref}
\hypersetup{
  colorlinks,
  citecolor=cyan,
  filecolor=blue,
  linkcolor=blue,
  urlcolor=magenta
}
\begin{document}
\begin{center}
  \thispagestyle{empty}
  \textbf{\Large{Installation Guide for \texttt{ByoDyn} version 4.8}}\\[5ex]
  \textbf{Adri\'an L\'opez Garc\'ia de Lomana, Alex G\'omez-Garrido, Miguel Hern\'andez, Pau Ru\'e Queralt and Jordi Vill\`a-Freixa}\\[10ex] --- December 17, 2008 ---\\[10ex]
  \parbox{0.70\linewidth}{
    This document specifies the requirements and different steps to install \texttt{ByoDyn} on a Linux box (Fedora Core 2, 4 and 6). 
  }\\[60ex] 
\end{center}
\begin{footnotesize}
  \begin{raggedleft}
    \textbf{Computational Biochemistry and Biophysics Laboratory}\\
    Research Unit on Biomedical Informatics\\
    Universitat Pompeu Fabra - Institut Municipal d'Investigaci\'o M\`edica\\
    c/ Dr. Aiguader 88, 08003, Barcelona, Spain\\
    \url{http://cbbl.imim.es}\\
  \end{raggedleft}
\end{footnotesize}
\newpage
\tableofcontents
\newpage
\section{\texttt{ByoDyn} Requirements} \label{requirements}
\subsection{Mandatory} \label{mandatory}
\begin{itemize}
\item
  \textbf{Python}: version 2.5 or newer is required \url{http://www.python.org/download}.
\item
  \textbf{SciPy}: version 0.6.0 or newer available at \url{http://www.scipy.org/}.
  To install SciPy you also need \texttt{numpy} available at \url{http://numeric.scipy.org/}.
  Please install \texttt{numpy} version 1.1.0 or newer.
  For an smooth use of SciPy, make sure you have built BLAS and LAPACK libraries as explained at \url{http://www.scipy.org/Installing_SciPy/BuildingGeneral}.
\item
  \textbf{libSBML}: version 3.2.0 or newer available at \url{http://www.sbml.org/libsbml.html}.
  Make sure you use at least the Python binder adding the flag \texttt{--with-python} while configuring.
\item
  \textbf{PORT Library}: from Netlib. 
  Please follow the requiered specified steps for the correct functioning of the local search routines of \texttt{ByoDyn}:
  \begin{itemize}
  \item 
    Use the following command to download the code:\\\texttt{rsync -avz netlib.org::netlib/port .}
  \item
    Modify the \texttt{Makefile}:
    \begin{itemize}
    \item
      If your Fortran compiler is not \texttt{f77} add the following line at the top the file:\\[1.5ex]
      \texttt{FC=gfortran}\\[1.5ex]
      \texttt{gfortran} is the default Fortran compiler in the latest Fedora distribution.
      If you have another compiler, change \texttt{gfortran} by the one you have.
    \item
      Change the line\\[1.5ex]
      \texttt{LIB=portP}\\[1.5ex]
      by \\[1.5ex]
      \texttt{LIB=port}\\[1.5ex]
      we are simply changing the name of the library.
    \item
      Remove the line\\[1.5ex]
      \texttt{n5err.o$\backslash$}\\[1.5ex]
      the file \texttt{n5err.f} was not found in the server and is not required in this case
    \item
      Change the line\\[1.5ex]
      \texttt{update lib\$(LIB).a \$?}\\[1.5ex]
      by\\[1.5ex]
      \texttt{ar rcs lib\$(LIB).a \$?}
    \item
      Change the line\\[1.5ex]
      \texttt{FFLAGS=-O}\\[1.5ex]
      by\\[1.5ex]
      \texttt{FFLAGS=-O -fPIC}\\[1.5ex]
      if your architecture is 64 bits.
    \end{itemize}
  \item
    Execute \texttt{make}.
  \end{itemize}
  Once you compile the source code, the file called \texttt{libport.a} has to be copied to the \\\texttt{byodyn/lib/local\_search/PORT} directory.
\item
  \textbf{Gnuplot}: Gnuplot 3.7 or newer needed to create the output graphs. 
  Available at \url{http://www.gnuplot.info/}.
\item
  \textbf{matplotlib}: Version 0.91.1 or newer is required.
  You can download it from \url{http://matplotlib.sourceforge.net/}.
  It is required for the plots output by \texttt{ByoDyn}.\\
\end{itemize}
\subsection{Optional}
In addition to this, other pieces of software add extra functionality to the \texttt{ByoDyn} platform:
\begin{itemize}
\item
  \textbf{Doxygen:} required to build the API of \texttt{ByoDyn} in HTML and PDF formats.
  It has been tested under Doxygen version 1.5.2.
\item
  \textbf{Octave}: is an optional tool to substitute SciPy functions for the integration of systems of ordinary differential equations (ODEs).
  On the contrary Octave is required for the solution of differential-algebraic equations (DAEs).
  Moreover Octave is compulsory required for clustering.
  Octave version 3.0.1 or newer is required.
  Octave is available at \url{http://www.octave.org/}.  
\item
  \textbf{OpenModelica}: version 1.4.3 or newer is required for the resolution of ODEs, DAEs and DAEs with events.
  Information for the installation can be found at \url{http://www.ida.liu.se/~pelab/modelica/OpenModelica.html}.
  Some environmental variables are required:
  \begin{itemize}
  \item \texttt{OPENMODELICAHOME}: pointing to the the \texttt{openmodelica} directory.
  \item \texttt{PATH}: \texttt{omc} and \texttt{./} should be on the \texttt{PATH} for a correct execution of the software.
  \end{itemize}
\item
  \textbf{XPP/XPPAUT}: version 5.91 or newer is required for solving SBML models with delays.
  Information for the installation can be found at \url{http://www.math.pitt.edu/~bard/xpp/xpp.html}.
\item
  \textbf{Open MPI}: the command \texttt{mpirun} is required to run \texttt{ByoDyn} in parallel.
  Version 1.2.3 or higher should be available.
\item
  \textbf{ScientificPython}: the module \texttt{Scientific.MPI} is required for launching parallel calculations.
  Version 2.6 or newer is required.
  Please check \url{http://dirac.cnrs-orleans.fr/plone/software/scientificpython/} for further information.
  Be sure you also create and call the executable \texttt{mpipython}.
  To build it please follow the instructions described in the \texttt{README.MPI} file of Scientific.
  If you will run \texttt{ByoDyn} on a single processor machine, the installation of this software is not required.
\end{itemize}
\section{Installation of \texttt{ByoDyn}}
Once all those software pieces have been installed into the machine and are accessible by \texttt{ByoDyn} you should set up a few things before you run it.
\begin{itemize}
\item
  First, add the \texttt{ByoDyn} executable to your path. 
  For bash this would be done by:
  \begin{itemize}
  \item
    edit your \texttt{\$HOME/.bashrc}.
  \item
    define the \texttt{ByoDyn} path: \texttt{export BYODYN\_PATH=\$HOME/where\_ByoDyn\_is}\\
    For example if you have \texttt{ByoDyn} at \texttt{\$HOME/a\_directory/another\_directory} you should add the line\\
    \texttt{export BYODYN\_PATH=\$HOME/a\_directory/another\_directory/byodyn}\\
    at the \texttt{.bashrc} file.
  \item
    Add \texttt{ByoDyn} executable file to the path: \texttt{export PATH="\$BYODYN\_PATH/bin:\$PATH"}
  \end{itemize}
\item
  Make sure that the file \texttt{libport.a} is at the correct directory as specified at the Section \ref{mandatory}.
\item
  Finally, execute the command \texttt{byodyn}.
  The following lines should prompt: \\
  \begin{footnotesize}
    \texttt{ByoDyn version 4.6 is running ...\\
      Run byodyn -h for help.\\
      ... exiting from ByoDyn.}
  \end{footnotesize}
\end{itemize}
At the case of a newly installed \texttt{ByoDyn}, we suggest to run the \texttt{ByoDyn} tests in order check the correct functioning of the program.
For that, type \texttt{byodyn --testing} on your terminal.
\end{document}
